\documentclass[11pt,a4paper]{ivoa}
\input tthdefs
\input gitmeta

\usepackage{hyperref}
\usepackage{verbatim}
\lstloadlanguages{XML,SQL}
\lstset{flexiblecolumns=true,numberstyle=\small,showstringspaces=False,
  identifierstyle=\texttt,basicstyle=\footnotesize}

\title{Sky Event Reporting Metadata (VOEvent)}

% see ivoatexDoc for what group names to use here
\ivoagroup[IG]{Time Domain}

\author[https://orcid.org/0000-0001-7915-5571]{Baptiste \textbf{Cecconi}, Observatoire de Paris-PSL, France}
\author[https://orcid.org/0000-0002-3709-8393]{\\Alasdair \textbf{Allan}, University of Exeter, UK}
\author{\\Scott \textbf{Barthelmy}, NASA Goddard Spaceflight Center, USA}
\author[https://orcid.org/0000-0002-7777-216X]{\\Joshua S. \textbf{Bloom}, University of California, Berkeley, USA}
\author[https://orcid.org/0000-0002-9873-1471]{\\John M. \textbf{Brewer}, Yale University, USA}
\author{\\Markus \textbf{Demleitner}, Universit\"at Heidelberg, Germany}
\author{\\Robert B. \textbf{Denny}, DC-3 Dreams SP, USA}
\author{\\Mike \textbf{Fitzpatrick}, National Optical Astronomy Observatory, USA}
\author{\\Matthew \textbf{Graham}, California Institute of Technology, USA}
\author[https://orcid.org/0000-0002-1941-9202]{\\Norman \textbf{Gray}, University of Glasgow, UK}
\author[https://orcid.org/0000-0002-0672-4945]{\\Frederic \textbf{Hessman}, University of Gottingen, Germany}
\author[https://orcid.org/0000-0001-9629-2922]{\\Pierre \textbf{Le Sidaner}, Observatoire de Paris, France}
\author[https://orcid.org/0000-0002-3957-1324]{\\Szabolcs \textbf{Marka}, Columbia University, USA}
\author[https://orcid.org/0000-0001-6847-2328]{\\Dave \textbf{Morris}, University of Edinburgh, UK}
\author[https://orcid.org/0000-0003-2377-2356]{\\Arnold \textbf{Rots}, Harvard-Smithsonian Center for Astrophysics, USA}
\author[https://orcid.org/0000-0001-9385-8978]{\\Rob \textbf{Seaman}, National Optical Astronomy Observatory, USA}
\author{\\Tom \textbf{Vestrand}, Los Alamos National Laboratory, USA}
\author[https://orcid.org/0000-0002-9145-8580]{\\Roy \textbf{Williams}, California Institute of Technology, USA}
\author[https://orcid.org/0000-0002-9919-3310]{\\Przemyslaw \textbf{Wozniak}, Los Alamos National Laboratory, USA}

\editor[mailto:baptiste.cecconi@obspm.fr]{Baptiste Cecconi}
\editor[mailto:seaman@noao.edu]{Rob Seaman}
\editor[mailto:roy@caltech.edu]{Roy Williams}

\previousversion[http://www.ivoa.net/documents/VOEvent/20110711/]{VOEvent 2.0}

\setcounter{secnumdepth}{5}

\begin{document}
\begin{abstract}
VOEvent defines the content and meaning of a
standard information packet for representing, transmitting, publishing and
archiving information about a transient celestial event, with the implication
that timely follow-up is of interest. The objective is to motivate the
observation of targets-of-opportunity, to drive robotic telescopes, to trigger
archive searches, and to alert the community. VOEvent is focused on the
reporting of photon events, but events mediated by disparate phenomena such as
neutrinos, gravitational waves, and solar or atmospheric particle bursts may
also be reported.

Structured data is used, rather than natural language, so that automated systems
can effectively interpret VOEvent packets. Each packet may contain zero or more
of the ``who, what, where, when \& how'' of a detected event, but in addition,
may contain a hypothesis (a ``why'') regarding the nature of the underlying
physical cause of the event.

Citations to previous VOEvents may be used to place each event in its correct
context. Proper curation is encouraged throughout each event's life cycle from
discovery through successive follow-ups.

VOEvent packets gain persistent identifiers and are typically stored in
databases reached via registries. VOEvent packets may therefore reference other
packets in various ways. Packets are encouraged to be small and to be processed
quickly. This standard does not define a transport layer or the design of
clients, repositories, publishers or brokers; it does not cover policy issues
such as who can publish, who can build a registry of events, who can subscribe
to a particular registry, nor the intellectual property issues.
\end{abstract}

\section{Introduction}

Throughout human history, unexpected events in the sky have been interpreted as
portents and revelations. Modern curiosity seeks to use such transient events to
probe the fundamental nature of the universe. In decades to come the scientific
study of such events will be greatly extended, with new survey telescopes making
wide-area systematic searches for time-varying astronomical events, and with a
large number of robotic facilities standing ready to respond. These events may
reflect purely local solar system phenomena such as comets, solar flares,
asteroids and Kuiper Belt Objects, or those more distant such as gravitational
microlensing, supernovae and Gamma-Ray Bursts (GRBs). Most exciting of all may
be new and unknown types of event, heralding new horizons for astrophysics.
Searches for astrophysical events are taking place at all electromagnetic
wavelengths from gamma-rays to radio, as well as quests for more exotic events
conveyed by such means as neutrinos, gravitational waves or high-energy cosmic
rays.

\subsection{Astrophysical events}

For many types of events, astrophysical knowledge is gained through fast,
comprehensive follow-up observation --- perhaps the immediate acquisition of the
spectrum of a suspected optical counterpart, for example --- and in general, by
observations made with instruments in different wavelength regimes or at
different times. To satisfy these needs, several projects are commissioning
robotic telescopes to respond to digital alerts by pointing the telescope and
triggering observations in near real-time and without human intervention. These
include, for instance, SkyAlert \citep{bib05} in the USA, and RoboNet-II
\citep{bib12} and eSTAR \citep{bib03} in the United Kingdom. Automated systems
may also query archives and initiate pipelines in response to such alerts.

Many projects have been conceived --- some now in operation --- that will
discover such time-critical celestial events. These include a large number of
robotic survey and monitoring telescopes with apertures from tens of centimetres
to tens of meters, large-field survey projects like Catalina Real-Time Transient
Survey (CRTS)\footnote{\url{http://crts.caltech.edu}} \citep{bib08}, Palomar
Transient Factory (PTF)\footnote{\url{https://www.ptf.caltech.edu}}
\citep{bib31}, Zwicky Transient Factory
(ZTF)\footnote{\url{https://www.ztf.caltech.edu}} \citep{2014htu..conf...27B},
Pan-STARRS\footnote{\url{https://panstarrs.stsci.edu}} \citep{bib09} and Large
Synoptic Survey Telescope (LSST)\footnote{\url{https://www.lsst.org}}
\citep{bib07}, satellites like Swift \citep{bib11a} and Fermi \citep{bib11b},
and more singular experiments like Laser Interferometer Gravitational Waves
Observatory (LIGO)\footnote{\url{https://www.ligo.org}} \citep{bib06}. The
community has demonstrated that robotic
telescopes\footnote{\url{http://www.astro.physik.uni-goettingen.de/~hessman/MONET/links.html}}
can quickly follow-up events using the standard outlined in this document. In the
past, human-centric event alert systems have been very successful, including the
Central Bureau for Astronomical Telegrams
(CBAT)\footnote{\url{http://www.cbat.eps.harvard.edu/index.html}} and the
Astronomer's Telegram
(ATEL)\footnote{\url{http://www.astronomerstelegram.org}} \citep{bib01}, but
these systems use predominantly natural-language text to describe each event,
and do not have sophisticated selection criteria for subscribers. The GRB
Coordinates Network (GCN)\footnote{\url{http://gcn.gsfc.nasa.gov}} \citep{bib04}
reports one of the most fruitful event streams of current times, and its events
are transmitted very successfully for follow-up within seconds or minutes. With
VOEvent, we leverage the success of GCN by making it interoperable with other
producers of events, and by generalising its transport mechanisms.

A much larger rate of events can be expected as new facilities are commissioned
or more fully automated. These rates indicate events that must be handled by
machines, not humans. Subscribing agents must be able to automatically filter a
tractable number of events without missing any that may be key to achieving
their goals. In general, the number of pending events from a large-scale survey
telescope (such as LSST) that are above the horizon at a given observatory
during a given observing session may be orders of magnitude larger than a human
can sift through productively. Selection criteria will need to be quite precise
to usefully throttle the incoming event stream(s) --- say --- ``\emph{give me
all events in which a point source R-band magnitude increase of at least -2.0
was seen to occur in less than four hours, that are located within specified
molecular column density contours of a prioritised list of galactic star forming
regions}''. In practice the result of complex queries such as these will be
transmitted through intermediary ``brokers'' --- which will subscribe to
VOEvent-producing systems and provide filter services to client groups
(``subscribers'') via specialised VOEvents. Filtering will often be based on
coincidence (spatial or temporal) between multiple events. A gravitational wave
detector may produce a large number of candidate events, but the interesting ones
may be only those that register with multiple instruments.

A recent study \citep{2018arXiv181112680C} extended the usage of VOEvent to
reports or predictions of Solar System events. This assessment revealed the
following needs: (a) The capability to specify the only target name as a
location. It can be a planet, a satellite, a comet, a moon, a spacecraft, a
rover, etc. Standard names should be used here, such as IAU names for natural
bodies. (b) The capability to specify the location using a planetary body
reference frame, or in a reference frame related to an object in the Solar
System. (c) The capability to specify a time range, with a start and end time.

\subsection{Why VOEvent?}

Handling the anticipated event rates quickly and accurately will require alert
packets to be issued in a structured data format, not natural language. Such a
structured discovery alert --- and any follow-up packets --- will be referred
to as a VOEvent. VOEvent will rely on XML
schema\footnote{\url{https://www.w3.org/XML/Schema}} to provide the appropriate
structured syntax and semantics. These schemata may be specific to VOEvent or
may implement external schemata such as the IVOA's Space-Time Coordinate (STC)
metadata specification \citep{2007ivoa.spec.1030R}. Some of the VOEvent structure
is provided by this document, for example the meaning of the \verb|<Who>|
and \verb|<Date>| elements; however other structure is provided by the author
of the event stream, who might define, for example, what the \verb|<peak_energy>|
and \verb|<energy_variance>| parameters mean when supplied with one of those events.

VOEvent is a pragmatic effort that crosses the boundary between the Virtual
Observatory and the larger astronomical community. The results of astronomical
observations using real telescopes will be expressed using the IVOA VOEvent
standard and disseminated by registries and brokers within and outside the VO.
Each event that survives rigorous filtering can then be passed to other
telescopes to acquire follow-up observations that will confirm (or deny) the
original hypothesis as to the classification of the object(s) or processes that
generated that particular VOEvent in the first place. This must happen quickly
(often within seconds of the original VOEvent) and must minimise unnecessary
expenditures of either real or virtual resources.

VOEvent is \emph{transport neutral}, but deploying and operating a robust
general-purpose network of interoperating brokers has always been a
high-priority issue. Various special-purpose networks and prototype networks
for the global VOEventNet have been deployed and operated. See references under
SkyAlert \citep{bib05} and Transport \citep{bib33} for two options.

Following the Abstract and Introduction, this document contains a discussion of
appropriate VOEvent usage in \S\ref{sec:2}. Section \S\ref{sec:3} is the heart
of the document, conveying the semantics of a VOEvent packet. Explicit examples
of VOEvent packets are in \S\ref{sec:4}, and linked references in \S\ref{sec:5}.

\subsection{Role within the VO Architecture}

\begin{figure}[ht!]
\centering\includegraphics[width=0.9\textwidth]{role_diagram}
\caption{VOEvent standard in the VO architecture diagram}
\label{fig:diagram}
\end{figure}

VOEvent is an IVOA standard, which means that it fits into a rich matrix of
other IVOA standards. Figure \ref{fig:diagram} shows where VOEvent fits into
the broader IVOA architecture \citep{2021ivoa.spec.1101D}.

VOEvents inherit much of the structure and semantics of \textbf{VOTable}
\citep{2019ivoa.spec.1021O}, including the \textbf{UCD}
\citep{2018ivoa.spec.0527P} scheme for semantics of quantities and the VOUnits
standard \citep{2023ivoa.spec.1215G}. VOEvent takes space-time coordinates from
the \textbf{STC} \citep{2007ivoa.spec.1030R}, and it uses the URI semantics of the
IVOA \textbf{Vocabulary} 1.0 \citep{2009ivoa.spec.1007G} effort. IVOA 
\textbf{Identifiers} \citep{2016ivoa.spec.0523D} are used for events and their
parent streams and servers, and both these latter are described by IVOA
\textbf{Resource Metadata} \citep{2007ivoa.spec.0302H} and stored in the VO
Registry.

\section{Usage}
\label{sec:2}
This document defines the syntax and semantics of an alert packet known as
VOEvent \citep{2011ivoa.spec.0711S}. In this document, the word \emph{packet}
will refer to a single, syntactically complete, VOEvent alert or message,
however transmitted or stored. The transmission of such a packet announces that
an astronomical ``event'' has occurred, or provides information contingent on a
previous VOEvent through a citation mechanism. The packet may include
information regarding the ``who, what, where, when \& how'' of the event, and
may express ``why'' hypotheses regarding the physical cause of the observed
event and the likelihood of each of these hypotheses.

\subsection{Publishing VOEvent Packets}
\label{sec:2.1}
VOEvent packets express sky transient alerts. VOEvent users subscribe to the
types of alerts pertinent to their science goals. The following roles define the
interchange of VOEvent semantics:
\begin{itemize}
\item An \emph\textbf{Author} is anyone (or any organisation) creating scientific
content suitable for representation as a sky transient alert. An author will
typically register with the IVOA registry, so that the \verb|<Who>| element of
VOEvent packets can be small and reusable, expressing only the IVOA identifier
needed to retrieve the contact information for the author. An authoring
organisation or individual may often rely on autonomous systems to actually
create and transport the individual alert messages.
\item A \emph\textbf{Publisher} receives alerts from one-or-more authors, and
assigns a unique identifier to each resulting packet. Either the author or the
publisher generates the actual XML syntax of the event, but the publisher is
responsible for the validity of the packet relative to the VOEvent schema.
Publishers will register with the IVOA registry as described below.
\item A \emph\textbf{Repository} subscribes to (or is party to the original
creation of) one or more VOEvent streams, persists packets either permanently
or temporarily, and runs a service that allows clients to resolve identifiers
and apply complex queries to its holdings. A given packet had one Publisher but
may be held in more than one Repository. Public repositories will register with
the IVOA registry.
\item A \emph\textbf{Subscriber} is any entity that receives VOEvent packets for
whatever purpose. Subscribers can find out how to get certain types of events
by consulting the lists of publishers and repositories in the IVOA registry.
A subscription is a filter on the stream of events from a publisher: the
subscriber is notified whenever certain criteria are met. For example, the
filter may involve the curation part of the event (\emph{e.g., ``all events
published by the Swift spacecraft''}), its location (\emph{``anything in
M31''}), or it may reference the detailed metadata of the event itself
(\emph{``whenever the cosmic ray energy is greater than 3 TeV''}).
\item A Broker or Relay, also sometimes known as a Filter, is any combination
of the atomic roles of Publisher, Repository, or Subscriber that also offers
arbitrary application-level functionality. See the IVOA VOEvent Transport Protocol
Recommendation \citep{2017ivoa.spec.0320S} for further discussion.
\end{itemize}

\subsection{VO Identifiers (IVOIDs)}
\label{sec:ivoids}
VOEvent benefits from the IVOA identifiers \citep{2016ivoa.spec.0523D}
developed for the VO registry.  In this document, such
an identifier is called an \emph{IVOID}, that is, an \emph{International Virtual
Observatory Identifier}
It is required to begin with ``\texttt{ivo://}'', and
will stand reference particular packet or an entire stream.

A \emph{registered} VOEvent packet is one
that has a valid identifier --- meaning that a mechanism exists that can resolve
that identifier to the full VOEvent packet. VOEvent identifiers thus provide a
citation mechanism --- a way to express that, for instance, one VOEvent packet is a
\emph{follow-up} in some fashion of a previous packet.

\begin{admonition}{Note}
When it talked about identifiers of VOEvents and their streams, VOEvent
Version 2.0 mostly talked about ``IVORNs'' IVOA Resource Names, rather
than IVOIDs.  This term was found to be misleading during the
preparation of Identifiers 2.0 \citep{2016ivoa.spec.0523D} and was
deprecated in that document's section 1.1.  Therefore, since version 2.1
of this specifiation, it adopts the term IVOID in the running text as
well.  This does not change the normative content; normative content
\emph{would} be changed if we also updated the names of the
\xmlel{ivoid} attribute or the \xmlel{AuthorIVORN} or \xmlel{EventIVORN}
element names, which therefore retain the deprecated nomenclature.
\end{admonition}

Historically, these have taken the general form $$
\hbox{\nolinkurl{ivo://authorityID/resourceKey#local_ID}}, $$ while new
VOEvent identifiers should follow the regulations of section 4.1 of
citet{2016ivoa.spec.0523D}, i.e.,$$
\hbox{\nolinkurl{ivo://authorityID/resourceKey?local_ID}}.$$

VOEvent identifiers are indirect
references to metadata packets that may be found in a VOEvent
repository.
In a compliant, registered VOEvent stream, each event's id without the
query part must reference a VO Registry record containing a capability
with a \xmlel{standardID} of \nolinkurl{ivo://ivoa.net/std/voevent} as per
Sect.~\ref{sec:registering}.  That resouce describes the stream at
which the VOEvent originated.  Publishers providing archives of their
events should include more capabilities through which archived events
can be found; this could, for instance, be a web page or perhaps a cone
search service.  Publishers relying on third parties to archive their
events should provide IsServedBy relationships to these third-party
services.

This is a key point in understanding VOEvent identifiers: The Event
identifier contains the Stream identifier.  For example,
\nolinkurl{ivo://example.org/catot?100407}
points to a specific VOEvent with the local id 100407
coming from the stream called \nolinkurl{ivo://example.org/catot}.
The full, unparsed
IVOID will not resolve in
the global VO Registry; only the part without the query string will.
To resolve the full identifier, a separate archive has to be set up and
mentioned in the registry record that
\nolinkurl{ivo://example.org/catot} points to.

\subsection{Authentication and Authorisation}
\label{sec:2.3}
VOEvents provide a mechanism for alerting members of the astronomical community
to time-critical celestial phenomena. As a result of such an alert, significant
hardware, software and personnel assets of the community may be retargeted to
investigate those phenomena. The scientific and financial costs of such
retargeting may be large, but the potential scientific gains are larger. The
success of VOEvent --- and of observations of astronomical transients in general
--- depends on minimising both intentional and unintentional noise/spam
associated with this communications channel. All of the familiar internet
security worries apply to VOEvents. A discussion of these issues is available
under Authentication \citep{bib34} from both the VOEvent standpoint and for
comparison, general XML signatures.

\section{VOEvent Semantics}
\label{sec:3}
A VOEvent packet provides a general purpose mechanism for representing
information about transient astronomical events. However, not all VO data are
suitable for expression using VOEvent. The VOEvent schema
\citep{2011ivoa.spec.0711S} is as simple as practical to allow the minimal
representation of scientifically meaningful, time critical, events. VOEvent
also borrows other standard VO and astronomical schema, specifically STC for
space-time coordinates. The usual IVOA standards such as registries and UCD
identifiers are used. VOEvent has a strong interest in the development of
complete and robust astronomical ontologies, but must rely on pragmatic and
immediately useful prototypes of planned facilities.

By definition, a VOEvent packet contains a single XML \verb|<VOEvent>| element.
If multiple \verb|<VOEvent>| elements are jointly contained within a larger
document in some fashion, they should still be handled as separate alert
packets. A \verb|<VOEvent>| element may contain at most one of each of the
following optional sub-elements:
\begin{itemize}
\item[\tt <Who>] Identification of scientifically responsible Author (see
\S\ref{sec:3.2})
\item[\tt <What>] Event Characterisation modelled by the Author (see
\S\ref{sec:3.3})
\item[\tt <WhereWhen>] Space-Time Coordinates of the event (see \S\ref{sec:3.4})
\item[\tt <How>] Instrument Configuration (see \S\ref{sec:3.5})
\item[\tt <Why>] Initial Scientific Assessment (see \S\ref{sec:3.6})
\item[\tt <Citations>] Follow-up Observations (see \S\ref{sec:3.7})
\item[\tt <Description>] Human Oriented Content (see \S\ref{sec:3.8})
\item[\tt <Reference>] External Content (see \S\ref{sec:3.9})
\end{itemize}

Only those elements required to convey the event being described need be
present; the ordering of elements is not formally constrained. The intent of
VOEvent is to describe a single astronomical transient event per packet.
Multiple events should be expressed using multiple packets. On the other hand,
complex observations may best be expressed using multiple follow-up packets or
via embedded \verb|<References>| to external resources such as VOTables or HTML
documents. XML structures other than those listed in this document should be
used with care within a \verb|<VOEvent>| element, but some applications may
require the freedom to reference schema outside the scope of this specification.
Section 4 contains examples of complete VOEvent packets.

\subsection{\texttt{<VOEvent>} --- identifiers, roles and versions}
\label{sec:3.1}
A \verb|<VOEvent>| expresses the discovery of a sky transient event, located in a
region of space and time, observed by an instrument, and published by a person
or institution who may have developed a hypothesis about the underlying
classification of the event.

The \verb|<VOEvent>| element has three attributes:

\noindent {\bf 3.1.1} \texttt{ivorn}\footnote{See sect.~\ref{sec:ivoids}
for why the attribute is called ivorn rather than ivoid; the attribute
name cannot be changed without breaking implementations.} \label{sec:3.1.1} ---
Each VOEvent packet is required to have one-and-only-one identifier, expressed
with the \xmlel{ivorn} attribute. VOEvent identifiers are URIs
\citep{2016ivoa.spec.0523D}. As the issuance of duplicate identifiers would
diminish the trust placed in systems exchanging VOEvents, it is anticipated that
a number of VOEvent publishers will be founded to issue unique IVOIDs from a
variety of useful and appropriate namespaces. The non-opaque URI identifier is
constructed systematically so that the identifier of a different resource, the
VOEventStreamRegExt, is deducible from the identifier of an event. The first
part is the identifier for the publisher, and the event identifier is built
from this, then a \texttt{\#} symbol, then a local string that is meaningful only
in the context of that publisher.

\noindent \textbf{3.1.2} \xmlel{role} \label{sec:3.1.2} ---
The optional \xmlel{role} attribute accepts the enumerated options:
\begin{itemize}
\item The value ``\emph{observation}'' is the default if the role is missing;
this means that the packet describes an observation of the actual universe.
\item The value ``\emph{prediction}'' indicates that the VOEvent describes an
event of whatever description that has yet to occur when the packet is created.
\item The value ``\emph{utility}'' means that the packet expresses nothing about
astrophysics, but rather information about the observing system. This could be
used, for example, for a satellite to express that it has changed its
configuration.
\item The value ``\emph{test}'' means that the packet does not describe actual
astronomical events, but rather is part of a testing procedure of some kind.
\end{itemize}
It is the responsibility of all who receive VOEvent packets to pay attention to
the \xmlel{role}, and to be quite sure of the difference between an actual event
and a test of the system or a prediction of an event that has yet to happen.

\noindent \textbf{3.1.3} \xmlel{version} \label{sec:3.1.3} ---
The \xmlel{version} attribute is required to be present and to equal "2.1" for
all VOEvent packets governed by this version of the standard. There is no
default value.

For example, a \verb|<VOEvent>| packet resulting from Tycho Brahe's discovery of
a ``Stella Nova'' in Cassiopeia on 11 November 1572 might start:
\begin{lstlisting}[language=XML]
<VOEvent ivorn="ivo://uraniborg.hven#1572-11-11/0001"
    role="observation" version="2.1">...
\end{lstlisting}

\subsection{\texttt{<Who>} --- Curation Metadata}
\label{sec:3.2}
This element of a VOEvent packet is devoted to curation metadata: who is
responsible for the information content of the packet. Usage should be
compatible with section 3.2 of the IVOA Resource Metadata specification
\citep{2007ivoa.spec.0302H}. Typical curation content would include:

\subsubsection{\texttt{<Author>}}
Author information follows the IVOA curation information schema: the
organisation responsible for the packet can have a title, short name or acronym,
and a logo. A contact person has a name, email, and phone number. Other
contributors can also be noted.

An example of Author information might be:
\begin{lstlisting}[language=XML]
<Author>
    <title>Rapid Telescope for Optical Response</title>
    <shortName>Raptor</shortName>
    <logoURL>http://www.raptor.lanl.gov/images/RAPTOR_patchLarge.jpg</logoURL>
    <contactName>Robert White</contactName>
    <contactEmail>rwhite@lanl.gov</contactEmail>
    <contactPhone>+1 800 555 1212</contactPhone>
 </Author>
\end{lstlisting}

A series of \verb|<Contributor>| (see below) elements can be added into the
\verb|<Author>| information.

\subsubsection{\texttt{<Contributor>}}
Contributor information can be included using as many \verb|<contributor>|
elements as necessary in \verb|Author|. The element value is the full name of the person or
organisation. Each element can have three optional attributes: an \xmlel{ivorn}
attribute to refer to the person's or organisation information in the VO
registry; an \xmlel{altIdentifier} attribute to refer to other identifier (such
as an ORCID (Open Researcher and Contributor ID)\footnote{\url{https://orcid.org}}
for persons, or a Research Organisation Registry
(ROR)\footnote{\url{https://ror.org}} identifier for institutions), in the form
of an URI; and a \xmlel{role} attribute. The \xmlel{role} attribute is an important
part of the contributor's metadata and allows proper attribution of work. We
recommend to use here the list of \emph{contributorType} from the DataCite
Metadata Schema v4.5 \citep{https://doi.org/10.14454/g8e5-6293}, as listed in
their documentation\footnote{\url{https://datacite-metadata-schema.readthedocs.io/en/4.5/appendices/appendix-1/contributorType/}}.

Here is an example of contributors in an \verb|<Author>|:
\begin{lstlisting}
<Author>
	<Contributor altIdentifier="https://orcid.org/0000-0001-7915-5571"
	    role="ContactPerson">Baptiste Cecconi</Contributor>
	<Contributor altIdentifier="https://ror.org/029nkcm90"
	    role="HostingInstitution">Observatoire de Paris</Contributor>
</Author>
\end{lstlisting}

\subsubsection{\texttt{<AuthorIVORN>}}
As an alternative to quoting Author information over and over, this information
can be published to the VO registry, then referenced through an IVORN.
The \verb|<AuthorIVORN>| element contains the identifier of the organisation responsible
for making the VOEvent available. Event subscribers will often use this as their
primary filtering criterion. Many subscribers will only want events from a
particular publisher, or more precisely, from a specific content creator. In
general, \verb|<AuthorIVORN>| should be a VOResource identifier that resolves to
an organisation in the sense of \citep{2007ivoa.spec.0302H}. Publishers and
subscribers may use this VOResource to exchange curation metadata directly.

\subsubsection{\texttt{<Date>}}
The \verb|<Date>| contains the date and time of the creation of the VOEvent
packet. The required format is a subset of ISO-8601 (\emph{e.g.,
\texttt{yyyy-mm-ddThh:mm:ss}}). The timescale --- for curation purposes only --- is
assumed to be Coordinated Universal Time (UTC). Discussions of date and time for
the expression of meaningful scientific coordinates may be found in
\citep{2007ivoa.spec.1030R} and \citep{bib26}.


Minimal \verb|<Who>| usage might resemble:
\begin{lstlisting}[language=XML]
<Who>
     <AuthorIVORN>ivo://uraniborg.hven/Tycho</AuthorIVORN>
     <Date>1573-05-05T01:23:45Z</Date>
</Who>
\end{lstlisting}
Tycho first noted SN 1572 on 11 November of that year. The event was published
in Tycho's pamphlet \emph{De Stella Nova} by 5 May 1573, thus this later date is
placed in the curation metadata. More detailed curation metadata can be
retrieved directly from the publisher.


\subsection{\texttt{<What>} --- Event Characterisation}
\label{sec:3.3}
The \verb|<What>| and \verb|<Why>| elements work together to characterise the
nature of a VOEvent. That is: \verb|<What>| has author-defined parameters about
what was measured directly, or other relevant information about the event,
versus \verb|<Why>| is a data model of fixed schema about the hypothesised
underlying cause or causes of the astrophysical event.

In general, an observation is the association of one or more dependent variables
with zero or more independent variables. The \verb|<WhereWhen>| element, for
example, is often used to express the independent variables in an observation
--- where was the telescope pointed and when was the camera shutter opened. The
\verb|<What>| element, on the other hand, is typically used to express the
dependent variables --- what was seen at that location at that time.

A \verb|<What>| element contains a list of \verb|<Param>| elements which may be
associated and labeled using \verb|<Group>| elements. It may also have one or
more \verb|<Table>| elements, each of which can contain \verb|<Param>| and \verb|<Field>|
elements: these last define a whole column, or vector of data, rather than a
single primitive value as with <Param>. See \S\ref{sec:4} for an example of
usage.

\subsubsection{\texttt{<Param>} --- Numbers and strings with semantics}
%\addtocounter{subsubsection}{1}
\label{sec:3.3.1}
\verb|<Param>| elements may be used to represent the values of arbitrarily named
quantities. Thus a publisher need not establish a fixed schema for all events
they issue. Unified Content Descriptors (UCDs) \citep{2018ivoa.spec.0527P}.
%\citep{std:UCD}
may be used to clarify meaning. Usage of \verb|<Param>| and \verb|<Group>| is
similar to the VOTable specification, see \S4.9 of \citep{2019ivoa.spec.1021O}.

A \verb|<Param>| may contain elements \verb|<Description>| and \verb|<Reference>|.
Like most VOEvent elements, these can be used to give further descriptive
documentation about what this parameter means. The \verb|<Param>| may also
contain an element \verb|<Value>| for the value of the parameter, as an alternate
to the `value' attribute defined below: if both are present, the attribute takes
precedence over the element. This allows parameter values to include a richer
variety of text strings, to avoid strings being changed by Attribute-Value
normalisation\footnote{\url{https://www.w3.org/TR/REC-xml/\#AVNormalize}} that
is part of the XML specification.

The following attributes are supported for \verb|<Param>|:

\noindent \textbf{3.3.1.1} \xmlel{name}\label{sec:3.3.1.1} --- A simple utilitarian
name. This name may or may not have significance to subscribing clients.

\noindent \textbf{3.3.1.2} \xmlel{value}\label{sec:3.3.1.2} --- A string representing
the value in question. No range or type checking of implied numbers is
performed.

\noindent \textbf{3.3.1.3} \xmlel{unit}\label{sec:3.3.1.3} --- The unit for
interpreting \xmlel{value}. See \S4.4 of \citep{2019ivoa.spec.1021O}
which relies on VOUnits \citep{2023ivoa.spec.1215G}.

\noindent \textbf{3.3.1.4} \xmlel{ucd}\label{sec:3.3.1.4} --- A UCD
\citep{2018ivoa.spec.0527P}
expression characterizing the nature of the \verb|<Param>|.

\noindent \textbf{3.3.1.5} \xmlel{dataType}\label{sec:3.3.1.5} --- A string specifying
the data type of the \verb|<Param>|. Allowed values are ``string'', ``int'', or
``float'', with the default being ``string''.
\begin{itemize}
\item For \xmlel{dataType=float}, the value must contain a possibly signed decimal
or floating point number, possibly embedded in whitespace; it may also be
$\pm$nan or $\pm$inf. If the value cannot be parsed this way, for example null
string, it may return zero or NaN, but no exception should be thrown.
\item For \xmlel{dataType=int}, the value must contain a possibly signed decimal
number, possibly embedded in whitespace. Conversion of floating point numbers to
integers truncates (towards zero). If the value cannot be parsed this way, for
example null string, it will return zero, and no exception should be thrown.
\end{itemize}

\noindent \textbf{3.3.1.6} \xmlel{utype}\label{sec:3.3.1.6} --- A \xmlel{utype} defines
this \verb|<Param>| as part of a larger data structure, such as one of the IVOA
standard data models. For more details, read the corresponding IVOA
page\footnote{\url{http://www.ivoa.net/cgi-bin/twiki/bin/view/IVOA/Utypes}}.

For example, here are three values from a GCN \citep{bib04} notice:
\begin{lstlisting}
TRIGGER_NUM = 114299 RATE_SIGNIF = 20.49 GRB_INTEN = 73288
\end{lstlisting}

In VOEvent, these can be represented as:
\begin{lstlisting}
<Param name="TRIGGER_NUM" value="114299" ucd="meta.id" />
<Param name="RATE_SIGNIF" value="20.49"  ucd="stat.snr" dataType="float">
    <Description>Best significance after trying all algorithms</Description>
    <Reference uri="http://gcn.gsfc.nasa.gov/swift.html"/>
  </Param>
<Param name="GRB_INTEN" value="73288"  ucd="phot.count" dataType="int"/>
\end{lstlisting}

\subsubsection{\texttt{<Group>} --- collection of related Params}
\label{sec:3.3.2}
\verb|<Group>| provides a simple mechanism for associating several \verb|<Param>|
(and/or \verb|<Reference>|) elements, for instance, an error with a measurement.
\verb|<Group>|s MUST NOT be nested. \verb|<Group>| elements may have a
\xmlel{name}
attribute, and unlike VOTable usage, may also have a \xmlel{type} attribute:

\noindent \textbf{3.3.2.1} \xmlel{name}\label{sec:3.3.2.1} --- A simple name such as
in \S\ref{sec:3.3.1.1}.

\noindent \textbf{3.3.2.2} \xmlel{type}\label{sec:3.3.2.2} --- A string that can be
used to build data structures, for example a Group with type ``complex'' might
have Params called ``real'' and ``imag'' for the two components of a complex
number.

In a GCN notice, for example, we might see this line:
\begin{lstlisting}
GRB_INTEN:       73288 [cnts]    Peak=1310 [cnts/sec]
\end{lstlisting}
which could be expressed with one Param with a Value element, and the other with
a Value attribute:
\begin{lstlisting}
<Group type="GRB_INTEN">
    <Param name="cnts" ucd="phot.count" dataType="int">
        <Value>73288</Value>
    </Param>
    <Param name="peak" value="1310" ucd="arith.rate;phot.count"
        dataType="float"/>
</Group>
\end{lstlisting}
Note also that there cannot be Groups within Groups: a Group may only contain
Params and not Groups or Tables; a Table may only contain Params and Fields and
not Groups or Tables. There are rules of uniqueness for Params, Groups, Fields
and Tables in VOEvent:
\begin{itemize}
\item Each Param and Field must have a name. A Group or Table without a name is
equivalent to having a name which is the null string.
\item Names must be unique within the set of those Params that are not in a
Group or Table.
\item Names must be unique for the set of Params and Fields within a given Group
or Table.
\item Groups and Tables must have unique names: this means that only one Group
or Table can be nameless.
\end{itemize}

\subsubsection{\texttt{<Table>} --- simple tabular data}
\label{sec:3.3.3}
This element is intended for a short and simple table, and re-uses the ideas and
syntax of the IVOA VOTable, but simplified and streamlined: this is appropriate
because complex tables can be written as full VOTable and linked from the
VOEvent. Specifically, these simplifications are: no support for hierarchy of
tables (RESOURCE); no internal references (FieldRef and ParamRef); no provision
for binary data, only XML; table cells can only be string, float, or int, in
place of the arrays of 12 possible types and extensions; no formatting
information contained in the Table, nor domain of the data (VALUES); no
referencing between cells; there is no INFO element.

There are five elements defined in this subsection: Table, Field, Data, TR, TD.

A \verb|<Table>| element can contain a sequence of \verb|<Field>| elements, one
for each column of the table, and \verb|<Param>| elements for scalar information
about the table. There is then a single \verb|<Data>| element that contains the
data of the table, which is represented as a sequence of table rows, which are
\verb|<TR>| elements, each containing a sequence of \verb|<TD>| elements for the
table cells. For a full table, where every cell has a value, the number
of \verb|<TD>| elements in each row must be the same as the number of \verb|<Field>|
elements. There is then a 1-to-1 correspondence between them for each row.

The Table can contain \verb|<Description>| and \verb|<Reference>| elements to add
documentation; the \verb|<Field>| elements can also contain these. Thus
the \verb|<Table>| can contain, in order, an optional \verb|<Description>| and 
\verb|<Reference>|, then a sequence of one or more \verb|<Field>| elements, then a
\verb|<Data>| element. The \verb|<Field>| element can also contain
optional \verb|<Description>| and \verb|<Reference>| and nothing else. The \verb|<Data>| element
can contain only \verb|<TR>| elements, each of which can contain only \verb|<TD>|
elements. The following explains the attributes that are allowed for these five
elements.

The following attributes are supported for \verb|<Table>|:

\noindent \textbf{3.3.3.1} \xmlel{name}\label{sec:3.3.3.1} --- A simple utilitarian
name that may be used for identification or presentation purposes. This name
may or may not have significance to subscribing clients.

\noindent \textbf{3.3.3.2} \xmlel{type}\label{sec:3.3.3.2} --- A string representing
the type of the Table, that consumers can use for presentation or parsing. For
example, a table of type ``spectralLines'' could mean to some community to
expect columns (i.e., the \verb|<Field>|s) named ``wavelength'', ``width'',
``name'' to define spectral lines.

The \verb|<Field>| element defines the semantic nature of a Table column, and is
structured similarly to the \verb|<Param>| element of section \ref{sec:3.3.1}.
The following attributes are supported for \verb|<Field>|, similarly to the
\verb|<Param>| definition above:

\noindent \textbf{3.3.3.3} \xmlel{name}\label{sec:3.3.3.3} --- A simple utilitarian
name that may be used elsewhere in the packet. This name may or may not have
significance to subscribing clients.

\noindent \textbf{3.3.3.4} \xmlel{unit}\label{sec:3.3.3.4} --- The unit for
interpreting the values as given in the \verb|<TD>| table cells. See \S4.4 of
\citep{2019ivoa.spec.1021O}, which relies on \citep{2023ivoa.spec.1215G}. 

\noindent \textbf{3.3.3.5} \xmlel{ucd}\label{sec:3.3.3.5} --- A UCD
\citep{2018ivoa.spec.0527P} expression characterizing the nature of the data in
this table column.

\noindent \textbf{3.3.3.6} \xmlel{dataType}\label{sec:3.3.3.6} --- A string specifying
the data type of the table column. Allowed values are ``string'', ``int'', or
``float'', with the default being ``string''.

\noindent \textbf{3.3.3.7} \xmlel{utype}\label{sec:3.3.3.7} --- A utype (see \S4.6 of
\citep{2019ivoa.spec.1021O}) defines this \verb|<Param>| as part of a larger data
structure, such as one of the IVOA standard data models.

 The following is an example of a Table element. Note the \xmlel{dataType}
 attribute that is used to interpret the values in the table cells.
\begin{lstlisting}[language=XML]
<Table>
    <Description>Individual Moduli and Distances for NGC 0931 from
      NED</Description>
    <Field name="(m-M)" unit="mag" ucd="phot.mag.distMod" dataType="float"/>
    <Field name="err(m-M)" unit="mag" ucd="stat.err;phot.mag.distMod"
      dataType="float"/>
    <Field name="D" unit="Mpc" ucd="pos.distance dataType="float"/>
    <Field name="REFCODE" ucd="meta.bib.bibcode"/>
    <Data>
        <TR><TD>33.16</TD><TD>0.38</TD><TD>42.9</TD>
          <TD>1997ApJS..109..333W</TD></TR>
        <TR><TD>33.32</TD><TD>0.38</TD><TD>46.1</TD>
          <TD>1997ApJS..109..333W</TD></TR>
        <TR><TD>33.51</TD><TD>0.48</TD><TD>50.4</TD>
          <TD>2009ApJS..182..474S</TD></TR>
    </Data>
</Table>
\end{lstlisting}

\subsection{\texttt{<WhereWhen>} --- Space-Time Coordinates}
\label{sec:3.4}

A VOEvent packet will typically include information about where in the sky and
when in time an event was detected, and from what location, along with spatial
and temporal coordinate systems and errors. If either the spatial or temporal
locators are absent, it is to be assumed that the information is either unknown
or irrelevant. VOEvent v2.1 borrows the syntax of the IVOA Space-Time Coordinate
(STC) specification version 1.30 or later; the \verb|<WhereWhen>| element may
reference an STC \citep{2007ivoa.spec.1030R} \verb|<ObsDataLocation>| element to
provide a sky location and time for the event. VOEvent publishers should
construct expressions that concisely provide all information that is
scientifically significant to the event, and no more than that. See
\S\ref{sec:4} for an example of usage.

STC expressions are used to locate the physical phenomena associated with a
VOEvent alert in both time and space as described below. The
\verb|<ObsDataLocation>| element is a combination of information describing the
location of an observation in the sky along with information describing the
location of the observatory from which that observation was made. Both the sky
and the observatory are in constant motion, and STC inextricably relates spatial
and temporal information.

\begin{lstlisting}[language=XML]
<WhereWhen>
    <ObsDataLocation>
        <ObservatoryLocation/>
        <ObservationLocation/>
    </ObsDataLocation>
 </WhereWhen>
\end{lstlisting}

\subsubsection{ObservationLocation}
\label{sec:3.4.1}

The \verb|<ObservationLocation>| defines the location of the event, whereas
the \verb|<ObservatoryLocation>| specifies the location of the observatory,
for which that event location is valid. It should contain a link to a
coordinate system, \verb|<AstroCoordSystem>|, as well as the actual coordinates
of the event, \verb|<AstroCoords>|, containing a reference back to the
coordinate system specification. For example:

\begin{lstlisting}
<ObservationLocation>
    <AstroCoordSystem id="UTC-FK5-GEO" />
    <AstroCoords coord_system_id="UTC-FK5-GEO">
        <Time unit="s">
            <TimeInstant>
                <ISOTime>2004-07-15T08:23:56</ISOTime>
            </TimeInstant>
            <Error>2</Error>
        </Time>
        <Position2D unit="deg">
            <Value2>
                <C1>148.88821</C1>
                <C2>69.06529</C2>
            </Value2>
            <Error2Radius>0.03</Error2Radius>
        </Position2D>
    </AstroCoords>
 </ObservationLocation>
\end{lstlisting}

Specifying errors is optional but recommended for both time and space
components.

The \verb|<AstroCoords>| element has a \xmlel{coord\_system\_id} attribute and the
\verb|<AstroCoordSystem>| has a \xmlel{id} attribute. The value of both of these
should be identical, and represent the space-time coordinate system that will be
used for the event position and time.

A \xmlel{coord\_system\_id} and \xmlel{id} are built from a time part, a space part,
and a ``center'' specification, concatenated in that order and separated by
hyphens. Astronomical coordinate systems are extremely varied, but all VOEvent
subscribers should be prepared to handle coordinates expressed as combinations
of these basic defaults:
\begin{itemize}
\item The time part can be \emph{UTC} (Coordinated Universal Time
\citep{bib26}), \emph{TT} (Terrestrial Time, currently 65.184 seconds ahead of
UTC), \emph{GPS} time, or \emph{TDB} (Barycentric Dynamical Time). The full list
of valid timescales is available as an IVOA vocabulary:
\url{https://www.ivoa.net/rdf/timescale}
\item The space part can be equatorial coordinates (right ascension and
declination) expressed in either the \emph{ICRS} or \emph{FK5} coordinate
systems. The list of valid reference frames is available as an IVOA vocabulary:
\url{https://www.ivoa.net/rdf/refframe}
\item The center specification can be \emph{TOPO} (i.e., the location of the
observatory), \emph{GEO} (geocentric coordinates), or \emph{BARY} (relative to
the barycenter of the solar system). Those terms are short versions of terms
listed in reference positions IVOA vocabulary: \url{https://www.ivoa.net/rdf/refposition}.
If the center specification is different from the three listed terms, the
\verb|<AstroCoordSystem>| must be defined explicitly, with the \verb|<TimeFrame>|
and \verb|<SpaceFrame>| elements.
\end{itemize}


It is assumed that the center reference position (origin) is the same for both
space and time coordinates. That means, for instance, that \emph{BARY} should
only be paired with \emph{TDB} (and vice-versa). See the STC specification
\citep{2007ivoa.spec.1030R} %\citep{std:STC}
for further discussion. The list of \verb|<AstroCoordSystem>| defaults that
VOEvent brokers and clients may be called upon to understand is: \\
\emph{TT-ICRS-TOPO, UTC-ICRS-TOPO, TT-FK5-TOPO, UTC-FK5-TOPO, GPS-ICRS-TOPO,
GPS-FK5-TOPO, TT-ICRS-GEO, UTC-ICRS-GEO, TT-FK5-GEO, UTC-FK5-GEO, GPS-ICRS-GEO,
GPS-FK5-GEO, TDB-ICRS-BARY, TDB-FK5-BARY}.

The STC specification, in particular \verb|<ObsDataLocation>| and its contained
elements, allows more exotic coordinate systems (for example, describing
planetary surfaces). Further description of how VOEvent packets might be
constructed to convey such information to subscribers is outside the scope of
this document. As with other elements of an alert packet, subscribers must be
prepared to understand coordinates expressing the science and experimental
design pertinent to the particular classes of sky transients that are of
interest.

In short, subscribers are responsible for choosing what VOEvent packets and thus
\xmlel{coord\_system\_id} values they will accept. Further, subscribers may choose
not to distinguish between coordinate systems that are only subtly different for
their purposes --- for instance between \emph{ICRS} or \emph{FK5}, or between
\emph{TOPO} or \emph{GEO}. As software determines whether a packet's
\xmlel{coord\_system\_id} describes a supported coordinate system, the question is also
what accuracy is required and what coordinate transformations may be implicitly
or explicitly performed to that level of accuracy.

A similar question faces the authors of VOEvent packets, who must make a
judicious choice between the available coordinate system options to meet the
expected scientific needs of consumers of those packets. If a detailed or high
accuracy coordinate system selection is not needed, \emph\textbf{UTC-ICRS-TOPO}
would be a good choice as an interoperability standard.

\subsubsection{ObservatoryLocation}
\label{sec:3.4.2}
The \verb|<ObservatoryLocation>| element is used to express the location from
which the observation being described was made. It is a required element for
expressing topocentric coordinate systems.

An instance of \verb|<ObservatoryLocation>| may take two forms. In the first,
an observatory location may be taken from a library, for example:
\begin{lstlisting}
<ObservatoryLocation id="Palomar" />
\end{lstlisting}

The \xmlel{id} here indicates the name of the observatory, other examples being:
Keck, KPNO, JCMT, MMTO, VLA, etc., or it may indicate one of the following
generic observatory locations:
\begin{itemize}
\item \emph{GEOSURFACE} - any location on the surface of the earth
\item \emph{GEOLEO} - any location in Low Earth Orbit (altitude<700 km)
\item \emph{GEOGSO} - any location within Geostationary orbit altitude
\item \emph{GEONBH} - any location within 50,000 km of the geocenter
\item \emph{GEOLUN} - any location within the Moon's orbit
\end{itemize}

For example, a packet might contain the following \verb|<ObservatoryLocation>|
to indicate that the coordinates expressed in the \verb|<WhereWhen>| element are
located with an accuracy comprising the Earth's surface:
\begin{lstlisting}[language=XML]
<ObservatoryLocation id="GEOSURFACE" />
\end{lstlisting}

The second option for \verb|<ObservatoryLocation>| is that an observatory can be
located by specifying the actual coordinate values of longitude, latitude and
altitude on the surface of the Earth. Note the use of a coordinate system for
the surface of the Earth (UTC-GEOD-TOPO) is natural for an observatory location,
whereas coordinate systems in the previous section are for astronomical events.
\begin{lstlisting}[language=XML]
<ObservatoryLocation id="KPNO">
    <AstroCoordSystem id="UTC-GEOD-TOPO" />
    <AstroCoords coord_system_id="UTC-GEOD-TOPO">
        <Position3D>
            <Value3>
                <C1 pos_unit="deg">248.4056</C1>
                <C2 pos_unit="deg">31.9586</C2>
                <C3 pos_unit="m">2158</C3>
            </Value3>
        </Position3D>
    </AstroCoords>
</ObservatoryLocation>
\end{lstlisting}

Each \xmlel{C1}, \xmlel{C2} and \xmlel{C3} element have
\xmlel{pos\_unit} and \xmlel{ucd}
optional attributes.

\subsubsection{\texttt{<AstroCoords>} --- Astronomy Coordinates}

As presented in the previous sections, the \verb|<AstroCoords>| has an
attribute \xmlel{coord\_system\_id} used to refer to the \verb|<AstroCoordSystem>|
in use for the coordinates values. Coordinates can be expressed with two
objects \verb|<Position2D>| or \verb|<Position3D>| depending on the dimensionality
of the coordinates, e.g., \verb|<Position2D>| for sky coordinates.

\paragraph{\texttt{<Position2D>}} is used to store 2D coordinate
positions. The names of the coordinate axes are set using \verb|<Name1>| and
\verb|<Name2>| elements. The values of the coordinates are set using a
\verb|<Value2D>|. The errors on the coordinate, if any, can be set using an
\verb|<Error2D>| object or an \verb|<Error2Radius>|.

\subparagraph{\texttt{<Value2D>}} is used to store the actual values of 2D
coordinates. It contains two elements: \verb|<C1>| and \verb|<C2>|, which are
both with type \xmlel{coord\_value}.

\subparagraph{\texttt{<Error2D>}} is used to store the uncertainty values of 2D
coordinates. It contains two elements: \verb|<C1>| and \verb|<C2>|, which are
both with type \xmlel{coord\_value}.

\subparagraph{\texttt{<Error2Radius>}} is used to store the uncertainty value
of 2D coordinates through a radius estimation. It has a type \xmlel{coord\_value}.

\paragraph{\texttt{<Position3D>}} is used to store 3D coordinate positions. The
names of the coordinate axes are set using \verb|<Name1>|, \verb|<Name2>| and
\verb|<Name3>| elements. The values of the coordinates are set using a
\verb|<Value3D>| object. The errors on the coordinate, if any, can be set using
a \verb|<Error3D>| object.

\subparagraph{\texttt{<Value3D>}} is used to store the actual values of 3D
coordinates. It contains three elements: \verb|<C1>|, \verb|<C2>| and
\verb|<C3>|, which are all with type \xmlel{coord\_value}.

\subparagraph{\texttt{<Error3D>}} is used to store the actual values of 3D
coordinates. It contains three elements: \verb|<C1>|, \verb|<C2>| and
\verb|<C3>|, which are all with type \xmlel{coord\_value}.

\paragraph{\xmlel{coord\_value}} is the object type of the coordinate
or uncertainty values. Such objects have two attributes \xmlel{pos\_unit} and
\xmlel{ucd}, which convey the units and UCD of the value.

\subsubsection{Parsing the WhereWhen Element}
\label{sec:3.4.3}
When parsing a VOEvent packet, the following pseudocode may be of use to provide
the time, the right ascension and the declination, if the author used
\emph{ICRS} spatial coordinates and \emph{UTC} time.
\begin{lstlisting}
Let  x =/voe:VOEvent/WhereWhen/ObsDataLocation/ObservationLocation/AstroCoords
   If x[@coord_system_id='UTC-ICRS-TOPO'] then
      Let Time = x/Time/TimeInstant/ISOTime
      Let RA = x/Position2D/Value2/C1
      Let Dec = x/Position2D/Value2/C2
\end{lstlisting}

The coordinate system is first checked to verify that it is set to a specific
value(s), \emph{UTC-ICRS-TOPO}. In practice, a subscriber may not care about the
difference between \emph{ICRS} and \emph{FK5} (of the order of 0.01'') or
between \emph{TOPO} and \emph{GEO} (in terms of timing, this is of the order of
25 ms for ground-based and low-earth-orbit observatories). Software may be
written to simply accept anything that contains \emph{ICRS} or \emph{FK5},
\emph{TOPO} or \emph{GEO}.


\subsubsection{Solar System Events}
\label{sec:3.4.4}
Solar system events include Solar events and planetary events.

Solar events have similar requirement as astronomical events in terms of
Observatory and Observation location but are using a different reference frames.
The following coordinate systems are recognised for solar event data:
\begin{itemize}
\item \emph{UTC-HPC-TOPO} --- Cartesian helio-projective coordinates (solar disk)
\item \emph{UTC-HPR-TOPO} --- Polar helio-projective coordinates (coronal events)
\item \emph{UTC-HGS-TOPO} --- Stonyhurst heliographic coordinates
\item \emph{UTC-HGC-TOPO} --- Carrington heliographic coordinates
\end{itemize}

These coordinate combinations shall be supported by VOEvent software (brokers
and clients) and that, hence, use of VOEvent by the solar research community is
supported. It does not imply, of course, that all VOEvent participants are
expected to recognise and handle these solar coordinates --- nor, for that
matter, that solar subscribers be able to handle equatorial coordinates.

Planetary events (including events at Earth, in a global solar system context,
e.g., for Space Weather or Near Earth Objects) have specific requirements that
have been discussed by \citet{2018arXiv181112680C}. Since many solar system body
reference frames exist, we do not list them here.

A \xmlel{PositionName} element is available in the
\xmlel{ObservationLocation/Ast\-roCoords} element. It is used to refer to named objects,
at which the event is observed without coordinates (e.g., for unresolved
observations, or global impact).

A \xmlel{TimeInterval} element is available in the
\xmlel{ObservationLocation/Ast\-roCoords/Time} element. It is composed of two elements
\xmlel{ISOTimeStart} and \xmlel{ISOTimeStop}, both defined similarly to
the \xmlel{ISOTime} element of \xmlel{TimeInstant}. This pair of dates is used to refer to
interval observations or predictions. This interval concept is different than
the error on the event Time, but rather corresponds to the boundaries of a
temporally extended event.

\subsubsection{Events Observed from Spacecraft}
\label{sec:3.4.5}
Transient event alerts resulting from observations made on distant spacecraft
may reference coordinates that require correction for ground-based follow-up.
The precise definition of ``distant'' will depend on the objects observed, the
instrumentation and the science program. For remote objects such as gamma-ray
bursts or supernovae, it is likely that spatial coordinates measured from
spacecraft in Earth orbit will be immediately useful --- indeed, the error box
of the reported coordinates may be much larger than that the pointing accuracy
of the follow-up telescope. On the other hand, the field of view of the
instrument on that telescope may be many times larger than the error box.
Subscribers must always balance such concerns --- this is just one facet of
matching ``scientific impedance'' between discovery and follow-up observations.

Even if the spatial targeting coordinates require no correction, the light
travel time may be quite significant between a spacecraft and any follow-up
telescopes on the Earth. Subscribers may need to adjust wavefront arrival times
to suit.

Authors of such events may choose to handle reporting the location of the
spacecraft in different ways. First, they may simply construct the
complex \verb|<ObservatoryLocation>| element that correctly represents the rapidly moving
location of an orbiting observatory. Further discussion of this topic is outside
the scope of the present document, see the STC specification
\citep{2007ivoa.spec.1030R}. Of course, any subscribers to such an event stream
would have to understand such an \verb|<ObservatoryLocation>| in detail and be
able to calculate appropriate time-varying adjustments to the coordinates in
support of their particular science program.

Alternately, an author of event alert packets resulting from spacecraft
observations might simply choose to correct their observations themselves into
geocentric or barycentric coordinates. Finally, for spacecraft in Earth orbit,
authors might choose to report an \verb|<ObservatoryLocation>| such as
\emph{GEOLUN}, indicating a rough position precise to the width of the Moon's
orbit. These two options might be combined by both making a geocentric
correction --- for instance, to simplify the handling of timing information ---
with the reporting of a \emph{GEOLEO} location, for example.

\subsection{\texttt{<How>} --- Instrument Configuration}
\label{sec:3.5}
The \verb|<How>| element supplies instrument specific information. A VOEvent
describes events in the sky, not events in the focal plane of a telescope. Only
specialised classes of event will benefit from providing detailed information
about instrumental or experimental design. A \verb|<How>| contains zero or more
\verb|<Reference>| elements (see \S\ref{sec:3.9}) and \verb|<Description>|
elements, that together characterise the instrument(s) that produced the
observation(s) that resulted in issuing the VOEvent packet. A URI pointing to a
previous VOEvent asserts that an identical instrumental configuration was used:
\begin{lstlisting}[language=XML]
<How>
    <Description> The Echelle spectrograph </Description>
    <Reference uri="http://nsa.noao.edu/kp012345.rtml" />
</How>
\end{lstlisting}

\subsection{\texttt{<Why>} --- Initial Scientific Assessment}
\label{sec:3.6}
\verb|<Why>| seeks to capture the emerging concept of the nature of the
astronomical objects and processes that generated the observations noted in the
\verb|<What>| is used to express the hypothesised astrophysics. Terms from
the IVOA UAT \citep{2022ivoa.spec.0722D} should be used here. Terms from other
controlled vocabularies may be used if necessary. Free text should only be used
for the cases where the relevant concepts are not described in existing vocabularies.

\subsubsection{Attributes}
The \verb|<Why>| element has two optional attributes, \xmlel{importance}
and \xmlel{expires}, providing ratings of the relative noteworthiness and
urgency of each VOEvent, respectively. Subscribers should consider the
\xmlel{importance} and \xmlel{expires} ratings from a particular publisher
in combination with other VOEvent
metadata in interpreting an alert for their purposes. The publishers of each
category of event are encouraged to develop a self-consistent rating scheme for
these values.

\paragraph{\xmlel{importance}}\label{sec:3.6.1}
The \xmlel{importance} provides a rating of the noteworthiness of the VOEvent,
expressed as a floating point number bounded between 0.0 and 1.0 (inclusive).
The meaning of \xmlel{importance} is unspecified other than that larger values are
considered of generally greater importance.

\paragraph{\xmlel{expires}}\label{sec:3.6.2}
The \xmlel{expires} attribute provides a rating of the urgency or time-criticality
of the VOEvent, expressed as an ISO-8601\footnote{\url{https://www.w3.org/TR/NOTE-datetime}}
representation of some date and time in
the future. The meaning of \xmlel{expires} is application dependent but will often
represent the date and time after which a follow-up observation might be
belated.

\subsubsection{Sub-elements}
A \verb|<Why>| element contains one or more \verb|<Concept>| and \verb|<Name>|
sub-elements. These may be used to assert concepts that specify a scientific
classification of the nature of the event, or rather to attach the name of some
specific astronomical object or feature. These may be organised using the
\verb|<Inference>| element, which permits expressing the nature of the
\xmlel{relation}
of the contained elements to the event in question as well as an estimate of its
likelihood via its \xmlel{probability} attribute.

\paragraph{\texttt{<Concept>} --- classification}\label{sec:3.6.3}
The value of a \verb|<Concept>| element uses terms from
the IVOA UAT \citep{2022ivoa.spec.0722D}. Terms from other controlled vocabularies
may be used if necessary. Free text should only be used for the cases where the
relevant concepts are not described in existing vocabularies.

\paragraph{\texttt{<Description>} --- natural language}\label{sec:3.6.4}
This element provides a natural language description of the concept, either as
a replacement for the \verb|<Concept>| element, or as an elaboration.

\paragraph{\texttt{<Name>} --- identification}\label{sec:3.6.5}
\verb|<Name>| provides the name of a specific astronomical object. It is
preferred, but not required, to use standard astronomical nomenclature,
\emph{e.g.}, as recognized by NED \citep{bib22} or SIMBAD \citep{bib23}.

\paragraph{\texttt{<Inference>} --- hypotheses inferred}\label{sec:3.6.6}
An \verb|<Inference>| may be used to group or associate one or more \verb|<Name>|
or \verb|<Concept>| elements. \verb|<Inference>| has two optional
attributes, \xmlel{probability} and \xmlel{relation}:
\begin{itemize}
\item \xmlel{probability}\label{sec:3.6.6.1} --- The
\xmlel{probability} attribute is an estimate of the likelihood of the \verb|<Inference>|
accurately describing the event in question. It is expressed as a floating point
number bounded between 0.0 and 1.0 (inclusive). In particular, note that
a \xmlel{probability} of 0.0 can be used to eliminate \verb|<Inferences>| from further
consideration.
\item \xmlel{relation}\label{sec:3.6.6.2} --- The \xmlel{relation}
attribute is a natural language string that expresses the degree of connection
between the \verb|<Inference>| and the event described by the packet. Typical
values might be ``associated'' --- a SN is associated with a particular galaxy
--- or ``identified'' --- a SN is identified as corresponding to a particular
precursor star. Such a one-to-one identification is considered to be the default
\xmlel{relation} in the absence of the attribute.
\end{itemize}

This example asserts that the creator of the packet is 100\% certain that the
event being described is equivalent to \emph{Tycho's Star}, which has been
identified as a \emph{Type Ia Supernova}, and is ``associated'' with the
\emph{SN remnant} known as \emph{3C 10}. This was an important discovery, but
is no longer a very urgent one:
\begin{lstlisting}
<Why importance="1.0" expires="1574-05-11T12:00:00">
    <Inference probability="1.0">
        <Name>Tycho's Stella Nova</Name>
        <Concept>https://ivoa.net/rdf/uat/#supernovae</Concept>
    </Inference>
    <Inference probability="1.0" relation="associated">
        <Name>3C 10</Name>
        <Concept>https://ivoa.net/rdf/uat/#supernova-remnants</Concept>
        <Description>Supernova remnant</Description>
    </Inference>
</Why>
\end{lstlisting}

\subsection{\texttt{<Citations>} --- Follow-up Observations}
\label{sec:3.7}
A VOEvent packet without a \verb|<Citations>| element can be assumed to be
asserting information about a new celestial discovery. Citations reference
previous events to do one of three things:
\begin{itemize}
\item follow-up an event alert with more observations or other relevant data, or
\item supersede a prior event with better, equivalent information, or
\item issue a complete retraction of a previous event.
\end{itemize}

Citations form the edges of a directed graph whose nodes are VOEvent instances;
they allow merging multiple events into a single related thread, a way to
collect multi-sourced data into a coherent whole. Projects that implement
VOEvent handling may decide to implement for different conditions of citation
--- perhaps assuming a sparse or structured citation graph, or a small or large
arity for each event. We recommend that the meaning of `citation' should be a
strong one: \emph{if a reader is to understand an event, then the reader should
understand the cited event}. This is the relation between a comment and a post,
between one observation of a transient and another relevant observation.
However, not everything should be cited: while the papers of Einstein may be
relevant, they need not be always cited! A different notion is that of
association of sources: as in a radio source being near an optical source. If an
author wishes to express this notion, the \verb|<Inference>| element can carry
this information (see section \ref{sec:3.6.6}).

A \verb|<Citations>| element contains one or more \verb|<EventIVORN>| elements.
The standard does not attempt to enforce references to be logically consistent;
this is the responsibility of publishers and subscribers.

\subsubsection{\texttt{<EventIVORN>} --- Cited event and relationship}
\label{sec:3.7.1}

An \verb|<EventIVORN>| element contains the IVORN of a previously published
VOEvent packet. Each \verb|<EventIVORN>| describes the relationship of the
current packet to that previous VOEvent. It has one required attribute:

\paragraph{\xmlel{cite}}\label{sec:3.7.1.1} --- The {cite} attribute accepts three
possible enumerated values, ``\emph{followup}'', ``\emph{supersedes}'' or
``\emph{retraction}''. There is no default value.

The value of the \xmlel{cite} attribute modifies the VOEvent semantics. In
contrast to a VOEvent announcing a discovery (\emph{i.e.}, a packet with no
citations), a VOEvent may be explicitly a ``\emph{followup}'', citing one or
more earlier packets --- meaning that the described real or virtual observation
was done as a response to those cited packet(s). In this case, the supplied
information is assumed to be a new, independent measurement.

The \xmlel{cite} may be ``\emph{supersedes}'', which can be used to express a
variety of possible event contingencies. A prior VOEvent may be superseded, for
example, if reprocessing of the original observation has resulted in different
values for quantities expressed by \verb|<What>| or \verb|<WhereWhen>| or if the
investigators have formed a new \verb|<Why>| regarding the event. On the other
hand, if a later observation has simply resulted in different measurements to
report, this would typically be issued as a ``\emph{followup}''.

When a citation is made with a ``\emph{supersedes}'' or ``\emph{retraction}''
attribute, it is assumed that \textbf{all} of the previous information is
superseded: and so the cited event is no longer needed other than for archival
or historical purposes. If there is datum X and datum Y in the original, and X
gets improved calibration, then Y must also be copied to the new event, or else
its value will no longer be seen. There is, however, no guarantee that a
superseded or retracted event will not be subsequently cited or referenced.

A ``\emph{supersedes}'' \xmlel{cite} can also be used to merge two or more earlier
VOEvent threads that are later determined to be related in some fashion. The
VOEvents to be merged are indicated with separate \verb|<EventIVORN>| elements.
The proper interpretation of such a merger would depend on a VOEvent client
having received all intervening packets from all relevant threads. Finally,
``\emph{supersedes}'' can be used in combination with a ``\emph{followup}'' to
divide a single VOEvent into two or more new threads. First, follow-up the event
in one packet and then supersede the original event, rather than the follow-up,
in a second packet (with a second identifier that can start a second thread).

The ``\emph{retraction}'' \xmlel{cite} indicates that the initial discovery event
is being completely retracted for some reason. The publisher of a retraction may
be other than the publisher of the original VOEvent --- subscribers are free to
interpret such a situation as they see fit.

Splitting, merging or retracting a VOEvent should typically be accompanied by a
\verb|<Description>| element discussing why such actions are being taken.

An attempt is made to retract the sighting of Tycho's supernova:
\begin{lstlisting}
<Citations>
    <EventIVORN 
        cite="retraction">ivo://uraniborg.hven#1572-11-11/0001</EventIVORN>
    <Description>Oops!</Description>
</Citations>
\end{lstlisting}

\subsection{\texttt{<Description>} --- Human Oriented Content}
\label{sec:3.8}
A \verb|<Description>| may be included within any element or sub-element of a
VOEvent to add human readable content. \verb|<Description>|s may NOT
contain
\verb|<References>|. Users may wish to embellish Description sections with HTML tags
such as images and URL links, and these should not be seen by the XML parser, as
they will cause the VOEvent XML to be invalid against the schema. However, it is
possible to use the CDATA mechanism of XML to quote text at length, so this may
be used for complicated tagged Descriptions. See the example in section
\ref{sec:4} for usage.

\subsection{\texttt{<Reference>} --- External Content}
\label{sec:3.9}

A \verb|<Reference>| may be included in any element or sub-element of a VOEvent
packet to describe an association with external content via a Uniform Resource
Identifier \citep{std:RFC3986}. In addition to the locator for the
content, there is also a locator for the meaning of the content, which is
another URI, specified by the \xmlel{meaning} attribute. It is anticipated that a
Note will be written discussing the IVOA-wide usage of such meaning locators. A
client application may ignore \verb|<Reference>| elements with
unrecognized \xmlel{meaning} attributes. On the other hand, the client may ignore the `meaning'
attribute if the position of the \verb|<Reference>| element is sufficient to
establish semantics; for example if it is contained in a \verb|<Param>|, then
presumably it gives drill-down semantics for the precise meaning of that
\verb|<Param>|. A \verb|<Reference>| must be expressed as an empty element, with
attributes only.

A \verb|<Reference>| element has the attributes:
\begin{itemize}
\item {\xmlel{uri}}\label{sec:3.9.1} --- The identifier of another document
(anyURI\footnote{\url{https://www.w3.org/TR/xmlschema11-2/\#anyURI}}). This
attribute must be present.
\item {\xmlel{meaning}}\label{sec:3.9.2} --- The nature of the document
referenced (anyURI). This attribute is optional.
\item {\xmlel{mimetype}}\label{sec:3.9.3} --- An optional RFC 2046 media
type of the referenced document \citep{std:MIME}.
\item {\xmlel{type}}\label{sec:3.9.4} [DEPRECATED] --- The type of the
document as described in VOEvent v1.11.
\item {\xmlel{name}}\label{sec:3.9.5} [DEPRECATED] --- A short name as
described in VOEvent v1.11.
\end{itemize}

A \verb|<Reference>| is used to provide general purpose ancillary data with
well-defined meaning. Here a fits image is presented (h.fits), and also a link
to the data model that is needed for a machine to understand the meaning.
\begin{lstlisting}[language=XML]
<Group type="MyFilterWithImage">
    <Reference uri=http://.../data/h.fits
        meaning="http://www.ivoa.net/rdf/IVOAT#Filter/h"/>
</Group>
\end{lstlisting}
An example of the indirection of a VOEvent packet using \verb|<Reference>|:
\begin{lstlisting}[language=XML]
<VOEvent ivorn="ivo://raptor.lanl#235649409/sn2005k"
    role="observation" version="2.0">
    <Reference uri="http://raptor.lanl.gov/documents/event233.xml"/>
</VOEvent>
\end{lstlisting}

\section{Event Streams and the Registry}
\label{sec:registry-matters}

In this section, we will reference several namespaced XML elements using
VO canonical prefixes.  The prefixes used here are:

\begin{itemize}
\item \verb|vr| -- \nolinkurl{http://www.ivoa.net/xml/VOResource/v1.0}
from \citet{2018ivoa.spec.0625P}.
\item \verb|vs| --
\nolinkurl{http://www.ivoa.net/xml/VODataService/v1.2}
from \citet{2021ivoa.spec.1102D}.
\item \verb|xsi| --
\nolinkurl{http://www.w3.org/2001/XMLSchema-instance}.
\end{itemize}

The canonical prefix for the VOEvent registry extension is \verb|voe|,
which maps to the namespace URI
\nolinkurl{http://www.ivoa.net/xml/VOEventRegExt/v2}.

\subsection{Registering Event Streams}
\label{sec:registering}

Public VOEvent streams MUST be registered in the VO
Registry\footnote{For a hands-on introduction on how to do that, refer
to
\url{https://wiki.ivoa.net/twiki/bin/view/IVOA/GettingIntoTheRegistry}}.
This is necessary to

\begin{compactitem}
\item ensure the validity of the event ids, as URIs with an \verb|ivo:|
scheme must resolve in a VO searchable registry to be valid
\citep{2016ivoa.spec.0523D}.
\item ensure the uniqueness of the event ids, as the event stream URI's
uniqueness is maintained using the Registry.
\item make the event streams enumeratable and findable.
\end{compactitem}

It is recommended to register VOEvent streams using
\xmlel{vs:CatalogService} (or, if the stream is only accessible through
third-party services, \xmlel{vs:Catalog\-Resource})
resources, as these allow service operators
to attach rich metadata like the originating facility and instrument, and
possibly extra stream metadata in a tableset.  However, this
specification does not constrain the resource type.

A public event stream MUST define a capability with standard id of
\nolinkurl{ivo://ivoa.net/std/VOEvent}.

Note that path parts in IVOA identifiers are case-insensitive, and hence
when comparing ivoids, clients must ignore case.

This specification does not constrain the type of the capability, but as
of this version, it is recommended to use plain \xmlel{vr:capability}
elements (i.e., not have \xmlel{xsi:type} attributes).

Zero or more endpoints publishing the event stream are declared within
this capability element using \xmlel{vr:interface} elements with their
\xmlel{role} attributes set to \verb|std|; such standard interfaces MUST
be of type \xmlel{voe:Stream\-Endpoint} and then by the schema MUST have
a \xmlel{standardID} attribute, the value of which SHOULD reference one
of the keys in this standard's registry record,
\nolinkurl{ivo://ivoa.net/std/VOEvent}.

As of this writing, these keys include:

\begin{compactitem}
\item  \verb|acc-vtp| The endpoint complies to IVOA VOEvent Transport
  Protocol \citep{2017ivoa.spec.0320S}
\item \verb|acc-xmpp| The endpoint uses an informal method based on
    	XMPP (jabber).
\item \verb|acc-kafka| The endpoint uses Apache Kafka \citep{kafka}.
\item \verb|acc-proprietary| The endpoint is usable by some
  method not (yet) mentioned in the VOEvent standard's registry record.
\end{compactitem}

New keys may be added to the registry record by consensus between the
chairs of the IVOA DAL and Time Domain Working groups.

Here is an example of a capability that will make a resource
discoverable as a VOEvent stream, with one endpoint each for VTP and
Kafka.  The Kafka stream is also availble through some other
provider, perhaps to ensure high availability:

\begin{lstlisting}
<capability standardID="ivo://ivoa.net/std/voevent">
  <interface xsi:type="voe:StreamEndpoint" role="std"
      standardID="ivo://ivoa.net/std/voevent#acc-vtp">
    <accessURL>http://example.org/events/vtp</accessURL>
  </interface>
  <interface xsi:type="voe:StreamEndpoint" role="std"
      standardID="ivo://ivoa.net/std/voevent#acc-vtp">
    <accessURL>http://example.org/events/kafka</accessURL>
    <accessURL>http://bigshot.com/streams/example-voe</accessURL>
  </interface>
</capability>
\end{lstlisting}

A full record describing a service running at the time of writing
comes with this
specification\footnote{\auxiliaryurl{resrec-sample.vor}}.  This also
shows how to declare an archive of the VOEvents sent out.

\subsection{Finding VOEvent Streams}

Normatively, VOEvent streams are located in the VO Registry as resources
with capabilites whose \xmlel{standardID} attribute compares equal to
the VOEvent standard id \nolinkurl{ivo://ivoa.net/std/voevent} ignoring
case.

This standard defines one mandatory details key for RegTAP
\citep{2019ivoa.spec.1011D}: \verb|capability/interface/standardID|.
Note that using this model, it is only possible to discover that a
service supports a given transport protocol but not to find out which
access URL corresponds to which transport protocol on multi-protocol
services.  If a consumer must make this distinction, it will need to
retrieve the service's capabilities document and parse it itself.

Using RegTAP, all registered VOEvent streams
can be located using a query like

\begin{lstlisting}[language=SQL]
SELECT ivoid, res_title
FROM rr.resource
  NATURAL JOIN rr.capability
WHERE standard_id='ivo://ivoa.net/std/voevent'
\end{lstlisting}

To get URLs and titles of streams giving a subject containing supernova
in some way, use a query like

\begin{lstlisting}[language=SQL]
SELECT res_title, access_url
FROM rr.resource
  NATURAL JOIN rr.capability
  NATURAL JOIN rr.interface
  NATURAL JOIN rr.res_subject
WHERE
  standard_id='ivo://ivoa.net/std/voevent'
  AND 1=ivo_nocasematch(res_subject, '%supernova%')
  AND role='std'
\end{lstlisting}

\section{VOEvent Examples}
\label{sec:4}
\subsection{Follow up observation of a supernova with the RAPTOR telescope}
This imaginary event is a brightness measurement of a past supernova from the
RAPTOR \citep{bib10} telescope. The \verb|<What>| section reports a
\verb|<Description>| and \verb|<Reference>| followed by a \verb|<Param>| about seeing
and a \verb|<Group>| with the actual report: the magnitude is 19.5, measured
278.02 days after the reference time, which is reported in the
\verb|<WhereWhen>| section. There is a \verb|<Table>| of measured distances to the
presumed host galaxy. The packet represents a follow-up observation of an
earlier event, as defined in the \verb|<Citations>| element.
\begin{lstlisting}[language=XML]
<?xml version="1.0" encoding="UTF-8"?>
<voe:VOEvent ivorn="ivo://raptor.lanl/VOEvent#235649409"
  role="observation"
  version="2.0"
  xmlns:xsi="http://www.w3.org/2001/XMLSchema-instance"
  xmlns:voe="http://www.ivoa.net/xml/VOEvent/v2.0"
  xsi:schemaLocation="http://www.ivoa.net/xml/VOEvent/v2.0
    http://www.ivoa.net/xml/VOEvent/VOEvent-v2.0.xsd">
  <Who>
    <AuthorIVORN>ivo://raptor.lanl/organization</AuthorIVORN>
    <Date>2005-04-15T14:34:16</Date>
  </Who>
  <What>
    <Description>An imaginary event report about SN 2009lw.</Description>
    <Reference uri="http://raptor.lanl.gov/data/lightcurves/235649409"
      mimetype="application/x-votable+xml"
      meaning="http://ivoa.net/rdf/uat#light-curves"/>
    <Param name="seeing" value="2" unit="arcsec"
      ucd="instr.obsty.seeing" dataType="float"/>
    <Group name="magnitude">
      <Description>Time is days since the ref time in the
        WhereWhen section</Description>
      <Param name="time" value="278.02" unit="d"
        ucd="time.epoch" dataType="float"/>
      <Param name="mag" value="19.5" unit="mag"
        ucd="phot.mag" dataType="float"/>
      <Param name="magerr" value="0.14" unit="mag"
        ucd="stat.err;phot.mag" dataType="float"/>
    </Group>
    <Table>
      <Param name="telescope" value="various"/>
      <Description>Individual Moduli and Distances for NGC 0931
        from NED</Description>
      <Field name="(m-M)" unit="mag" ucd="phot.mag.distMod"/>
      <Field name="err(m-M)" unit="mag" ucd="stat.err;phot.mag.distMod"/>
      <Field name="D" unit="Mpc" ucd="pos.distance"/>
      <Field name="REFCODE" ucd="meta.bib.bibcode"/>
      <Data>
        <TR><TD>33.16</TD><TD>0.38</TD><TD>51.3</TD><TD>1997ApJS..109..333W</TD></TR>
        <TR><TD>33.32</TD><TD>0.38</TD><TD>46.1</TD><TD>1997ApJS..109..333W</TD></TR>
        <TR><TD>33.51</TD><TD>0.48</TD><TD>50.4</TD><TD>2009ApJS..182..474S</TD></TR>
        <TR><TD>33.55</TD><TD>0.38</TD><TD>51.3</TD><TD>1997ApJS..109..333W</TD></TR>
        <TR><TD>33.71</TD><TD>0.43</TD><TD>55.2</TD><TD>2009ApJS..182..474S</TD></TR>
        <TR><TD>34.01</TD><TD>0.80</TD><TD>63.3</TD><TD>1997ApJS..109..333W</TD></TR>
      </Data>
    </Table>
  </What>
  <WhereWhen id="Raptor-2455100">
    <ObsDataLocation>
      <ObservatoryLocation id="RAPTOR"/>
      <ObservationLocation>
        <AstroCoordSystem id="UTC-ICRS-TOPO"/>
        <AstroCoords coord_system_id="UTC-ICRS-TOPO">
          <Time>
            <TimeInstant>
              <ISOTime>2009-09-25T12:00:00</ISOTime>
            </TimeInstant>
            <Error>0.0</Error>
          </Time>
          <Position2D unit="deg">
            <Value2>
              <C1>37.0603169</C1>
              <!-- RA  -->
              <C2>31.3116578</C2>
              <!-- Dec -->
            </Value2>
            <Error2Radius>0.03</Error2Radius>
          </Position2D>
        </AstroCoords>
      </ObservationLocation>
    </ObsDataLocation>
  </WhereWhen>
  <How>
    <Description>
      <![CDATA[This VOEvent packet resulted from observations made with
        <a href=http://www.raptor.lanl.gov>Raptor</a> AB at Los Alamos. ]]>
    </Description>
  </How>
  <Citations>
    <EventIVORN cite="followup">ivo://raptor.lanl/VOEvent#235649408</EventIVORN>
  </Citations>
  <Why>
    <Concept>http://ivoat.ivoa.net/process.variation.burst;em.opt</Concept>
    <Description>Looks like a SN</Description>
    <Inference relation="associated" probability="0.99">
      <Name>NGC0931</Name>
    </Inference>
  </Why>
</voe:VOEvent>
\end{lstlisting}

\subsection{Prediction of a Solar Wind event arrival time at Jupiter}
This second imaginary example describes the predicted time of arrival of
a Solar Wind dynamic pressure pulse at Jupiter. The prediction has been
produced by a 1D MHD propagation model \cite{tao05}, referred to as with
the DOI of the paper describing the code. The \xmlel{WhereWhen} section provides
the location of the event detection, and the time frame in use for the
predicted dates. The \xmlel{What} section provides the interval, in which the
detection threshold is met. The reference dataset is also cited in the
\xmlel{Why} section.
\begin{lstlisting}[language=XML]
<?xml version="1.0"?>
<voe:VOEvent ivorn="ivo://psws.irap/VOEvent/Tao_Jupiter_2018-10-02T17_34_45::v1.0"
  role="prediction" version="2.1"
  xmlns:xsi="http://www.w3.org/2001/XMLSchema-instance"
  xmlns:voe="http://www.ivoa.net/xml/VOEvent/v2.1"
  xsi:schemaLocation="http://www.ivoa.net/xml/VOEvent/v2.1
    http://www.ivoa.net/xml/VOEvent/VOEvent-v2.1.xsd">
  <Who>
    <AuthorIVORN>ivo://psws</AuthorIVORN>
    <Author>
      <contactEmail>Michel.Gangloff@irap.omp.eu</contactEmail>
      <contactName>Michel Gangloff</contactName>
    </Author>
    <Date>2018-10-02T17:34:45</Date>
  </Who>
  <What>
    <Description>Time intervals in which dynamic pressure is greater than
      0.08 nPa at Jupiter</Description>
    <Param name="event_type" value="Solar Wind dynamic pressure pulse" ucd="meta.id"/>
    <Group name="event">
      <Param name="tested_parameter" value="Dynamic Pressure" ucd="meta.id" dataType="string"/>
      <Param name="threshold" value="0.08" ucd="phys.pressure" unit="nPa" dataType="string"/>
    </Group>
    <Group name="target">
      <Param name="target_name" value="Jupiter" ucd="meta.id" utype="epn:target_name"/>
      <Param name="target_class" value="planet" ucd="meta.id" utype="epn:target_class"/>
      <Param name="target_region" value="Magnetosphere" ucd="meta.id" utype="epn:target_region"/>
    </Group>
    <Table>
      <Description>Time intervals found by the propagation code</Description>
      <Field dataType="string" name="Start Time" ucd="time.begin" utype="epn:time_min">
        <Description>time tag for beginning of interval</Description>
      </Field>
      <Field dataType="string" name="Stop Time" ucd="time.end" utype="epn:time_max">
        <Description>time tag for end of interval</Description>
      </Field>
      <Data>
        <TR><TD>2018-10-03T10:00:00.000</TD><TD>2018-10-04T01:00:00.000</TD></TR>
        <TR><TD>2018-10-04T07:00:00.000</TD><TD>2018-10-04T11:00:00.000</TD></TR>
        <TR><TD>2018-10-08T01:00:00.000</TD><TD>2018-10-08T07:00:00.000</TD></TR>
      </Data>
    </Table>
  </What>
  <WhereWhen>
    <ObsDataLocation>
      <ObservatoryLocation/>
      <ObservationLocation>
        <AstroCoordSystem>
          <TimeFrame>
            <ReferencePosition>Jupiter</ReferencePosition>
            <TimeScale>UTC</TimeScale>
          </TimeFrame>
        </AstroCoordSystem>
        <AstroCoords>
          <PositionName>Jupiter</PositionName>
        </AstroCoords>
      </ObservationLocation>
    </ObsDataLocation>
  </WhereWhen>
  <How>
    <Description>This prediction has been computed by the Heliospheric propagation 1D MHD model
      for solar wind prediction at planets, probes and comets.</Description>
    <Reference uri="https://doi.org/10.1029/2004JA010959"/>
  </How>
  <Why>
    <Inference probability="1.0">
      <Concept>http://astrothesaurus.org/uat/1534</Concept>
      <Name>Solar Wind measured at 1 AU</Name>
      <Reference uri="https://doi.org/10.48322/1shr-ht18"/>
    </Inference>
    <Inference>
      <Concept>http://astrothesaurus.org/uat/1966</Concept>
      <Name>MHD simulation of propagated Solar Wind at Jupiter</Name>
      <Reference uri="https://doi.org/10.1029/2004JA010959"/>
    </Inference>
  </Why>
  <Description>3 Solar Pressure pulses predicted at Jupiter between 2018-10-03 and 2018-10-09</Description>
</voe:VOEvent>
\end{lstlisting}

\section{Schema Diagram for VOEvent}
\label{sec:5}
This image summarizes the basic structure of the event packet. The image shows
how the \verb|<Description>| and \verb|<Reference>| elements can appear in many
different places, abbreviated by D and R. Elements and their hierarchy are in
black, attributes in green, required attributes underlined.
\begin{figure}[th]
\begin{center}
\includegraphics[width=0.9\textwidth]{nutshell.png} \end{center}
\caption{VOEvent version 2 in a Nutshell.}
\label{fig:nutshell}
\end{figure}


%\appendix

\bibliography{ivoatex/ivoabib,ivoatex/docrepo,localrefs}

\appendix

%%% move to end of document as required in IvoaTeX specification
%\section{Changes from Previous Versions}
\section{Modification History}
\subsection{Changes from VOEvent 2.0}
\label{appendix:last-changes}
\begin{itemize}
\item The \xmlel{contributor} element has new attributes: \xmlel{ivorn},
\xmlel{altIdentifier} and \xmlel{role}.
\item The restricted list of \xmlel{AstroCoordSystem} is removed. It was
previously an \xmlel{idValues} type, now it is a simple \xmlel{xs:string} type.
This allows to include Solar and Planetary frames without modifying the
schema. The \xmlel{idValues} type and its references (in \xmlel{AstroCoordSystem}
and \xmlel{coord\_system\_id}) have been removed. The \xmlel{AstroCoordSystem} can
be fully described with a \xmlel{TimeFrame} and a \xmlel{SpaceFrame}, see 
\citet{2007ivoa.spec.1030R}.
\item Annotations in \xmlel{AstroCoords/Time} and \xmlel{AstroCoords/Position2D}
have been included in the schema (according to STC-1.33).
\item The concept of \xmlel{AstroCoords/PositionName} is introduced with
type \xmlel{xs:string}. This allows to identify a target by its name (such as a named solar
system body).
\item The concept of \xmlel{TimeInterval} is introduced in the \verb|<Time>|
section. It contains a \xmlel{ISOStartTime} and a \xmlel{ISOStopTime}
\item The positional error elements have been improved. The \xmlel{Position2D/Er\-ror2Radius}
is now optional, and a new \xmlel{Position2D/Error2}
concept is introduced (allowing to describe error bars on each of the 2D frame
axes). A \xmlel{Position3D/Error3} concept is also introduced.
\item Each individual positional value are now associated with their own UCD
and Unit.
\end{itemize}

\subsection{Changes from VOEvent 1.11}
\begin{itemize}
\item The concept of event stream is introduced in section \ref{sec:ivoids}, this
is new in VOEvent 2.0. The stream metadata acts as a template for the events in
the stream, and is registered with the VO registry.
\item The section on transport of VOEvents has been removed, so it can be
handled in its own standards process.
\item The section on Registry enhancements to support VOEvent has been expanded
and clarified.
\item The \verb|<Param>| elements can now have \verb|<Description>| and
\verb|<Reference>| elements
\item The value of a \verb|<Param>| element can now be expressed as an element
in addition to an attribute.
\item The \verb|<Param>| element now has an attribute ``\xmlel{dataType}'' to
express the meaning of the parameter value (\texttt{int}, \texttt{float}, \texttt{string}).
\item There is a new \verb|<Table>| element to express simple tables, see section
\ref{sec:3.3.3}.
\item The \verb|<Param>| and \verb|<Field>| elements may have an attribute
``\xmlel{utype}'' to express how it fits into an IVOA data model.
\item The VOEvent packet structure still conforms to the IVOA Space-Time
Coordinates standard, but there is a new, simplified schema for these elements
that is completely within the VOEvent schema.
\item GPS time is now a valid time system for VOEvents
\item The semantic implication of a \verb|<Citation>| element is clarified:
section \ref{sec:3.7}
\item The \verb|<Reference>| element has a more sophisticated notion of meaning;
it is a general URI reference to a wide range of possible content, rather than
just a simple HTML link, and there is also a \xmlel{mimetype} attribute.
\end{itemize}

\section{Schema}
\label{sec7}
The XML schema available at
\url{http://www.ivoa.net/xml/VOEvent/VOEvent-v2.1.xsd} corresponds to this
document, but it is the document that is normative.
\end{document}
